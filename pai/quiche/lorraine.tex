\section*{Quiche Lorraine}


\subsection*{Dette trenger du}


\begin{table}[!htbp]
    %\centering
    \begin{tabular}[t]{rl}
        \textbf{1 stk}                  & paibunn (se \hyperref[pai:paibunn]{Paibunn})  \\
        \textbf{1 stk}                  & gulløk                                        \\
        \textbf{\SI{150}{\gram}}        & bacon                                         \\
        \textbf{4 stk}                  & egg                                           \\
        \textbf{\SI{2}{\deci\litre}}    & matfløte
    \end{tabular}
    \qquad
    \begin{tabular}[t]{rl}
        \textbf{\SI{1}{\deci\litre}}    & lettmelk $1.0 \%$ fett                        \\
        \textbf{\SI{100}{\gram}}        & revet gulost                                  \\
        \textbf{\sfrac{1}{2} ts}        & salt                                          \\
        \textbf{\sfrac{1}{4} ts}        & pepper                                        \\
        \textbf{\sfrac{1}{4} ts}        & revet muskatt 
    \end{tabular}
\end{table}



\subsection*{Slik gjør du}

\begin{enumerate}
    \item 
    Etter å ha laget paibunn og forhåndsstekt bunnen, sett ovnen til \SI{200}{\celsius} over og undervarme.
    
    \item 
    Skrell og kutt opp løken. 
    Del bacon i terninger eller skiver.
    
    \item 
    Stek løken på middels sterk varme i olje i 3-4 minutter, til den er blank.
    Stek deretter baconet til ønsket sprøhet.
    
    \item 
    Pisk sammen egg, fløte, melk, salt, pepper og muskattnøtt i en bolle.
    Tilsett den stekte løken og baconet i blandingen.
    
    \item
    Hell dette over i paibunnen og ha på revet ost. 
    
    \item
    Stek quichen på \SI{200}{\celsius} over og undervarme i 20-25 minutter til osten er gyldenbrun på toppen. 
    
\end{enumerate}