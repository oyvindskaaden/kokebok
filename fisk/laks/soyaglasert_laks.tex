\section*{Soyaglasert laks med ris (2 porsjoner)}

\subsection*{Dette trenger du}


\begin{table}[!htbp]
    %\centering
    \begin{tabular}[t]{rl}
        \textbf{300 g}      & laksefilet            \\
        \textbf{2 ss}       & olivenolje            \\
        \textbf{1,5 ss}     & akasiehonning         \\
        \textbf{4 ss}       & soyasaus              \\
        \textbf{1 ss}       & revet ingefær
    \end{tabular}
    \qquad
    \begin{tabular}[t]{rl}
        \textbf{1 fedd}     & hvitløk               \\
        \textbf{1 dl}       & vann                  \\
                            & sesamfrø etter ønske  \\
        \textbf{1 pakke}    & boil-in-bag ris
    \end{tabular}
    %\caption{Caption}
    %\label{tab:my_label}
\end{table}

\subsection*{Slik gjør du}

\begin{enumerate}
    \item 
    Sett ovnen på 180 grader
    
    \item
    Finhakk hvitløken og rasp ingefæren.
    
    \item 
    Bland olivenolje, akasiehonning, soyasaus, ingefær og hvitløk i en middels stor bolle.
    
    \item
    Skjær laksefileten i porsjonsstykker. Legg stykkene i soyamarinaden, og mariner i \textbf{10 minutter}. 
    
    Dersom du har nok tid, blir fisken bedre jo lenger du lar den trekke i sausen. Gjerne opp mot en time.
    
    \item
    Kok risen etter pakken.
    
    \item 
    Ta laksen opp av marinaden og ta vare på resten av marinaden. 
    Stek deretter laksestykkene i solsikkeolje på høy varme i cirka \textbf{1 minutt} på hver side. 
    Ha så laksestykkene i en ildfast form, strø på ønsket mengde sesamfrø og stek ferdig i ovnen i \textbf{5 minutter}.
    
    \item
    Resten av marinaden har du tilbake i stekepannen, sammen med \textbf{1 dl} vann og saften fra halve limen. 
    Kok ned til cirka halvparten.
    
    \item
    Når laksen er ferdigstekt i ovnen, glaserer du den med soyasausen fra punktet over. 
    
    \item
    Server sammen med den nykokte risen.
\end{enumerate}