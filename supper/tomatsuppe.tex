\section*{Tomatsuppe (4 porsjoner)}

\url{https://trinesmatblogg.no/recipe/hjemmelaget-tomatsuppe-2/}

\subsection*{Dette trenger du}


\begin{table}[!htbp]
    %\centering
    \begin{tabular}{rl}
        \textbf{2}                  & sjarlottløk            \\
        \textbf{2 fedd}             & hvitløk               \\
        \textbf{\sfrac{1}{2}}       & rød chilli            \\
        \textbf{2 ss}               & tomatpuré             \\
    \end{tabular}
    \qquad
    \begin{tabular}{rl}
        \textbf{2 bokser (800 g)}   & hermetiske tomater    \\
        \textbf{7,5 dl}             & grønnsaksbuljong      \\
                                    & sukker                \\
                                    & salt og pepper        \\
    \end{tabular}
\end{table}



\subsection*{Slik gjør du}

\begin{enumerate}
    \item 
    Finhakk sjarlottløk og hvitløken.
    
    \item 
    Fres løken på middels varme med nøytral matolje til løken er blank og myk. 
    Mot slutten tilsetter du hvitløk og chili, og lar det surre med. 
    
    \item 
    Tilsett tomatpureèn og la det surre med et par minutter.

    \item 
    Tilsett de hermetiske tomatene, knus dem lett med en stekespade eller lignende, og gi suppen et oppkok. Tilsett deretter kraften og kok opp igjen. 

    \item
    La suppen småkoke til tomatene er møre og fine. Det tar ca 15-20 minutter dersom du benytter hele tomater. Benytter du hakkede tomater kan koketiden gjerne reduseres til ca 10. minutter hvis du har det travelt eller er veldig sulten. 
    
    \item 
    Kjør suppen med en stavmikser eller lignende, til du får en jevn suppe. Tilsett eventuelt mer vann eller kraft hvis du synes suppen blir for tykk. Smak til med nykvernet sort pepper og salt.
    
\end{enumerate}