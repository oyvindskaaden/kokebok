\section*{Havrebrød (3 brød)}

\subsection*{Dette trenger du}

\begin{table}[!htbp]
    %\centering
    \begin{tabular}{rl}
        \textbf{1000 g}                 & hvetemel              \\
        \textbf{200 g}                  & sammalt hvete fin     \\
        \textbf{225 g}                  & sammalt hvete grov    \\
        \textbf{225 g}                  & sammalt rug fin       \\
        \textbf{1,4 l}                  & vann                  \\
    \end{tabular}
    \qquad
    \begin{tabular}{rl}
        \textbf{2 ss}                   & maltekstrakt          \\
        \textbf{285 g}                  & havregryn             \\
        \textbf{\sfrac{1}{2} pakke}     & tørrgjær              \\
        \textbf{30-40 g}                & havsalt               \\ \\
    \end{tabular}
\end{table}

\subsection*{Slik gjør du}

\begin{enumerate}
    \item 
    Bland sammen alt melet, gjæren og nesten alt vannet. 
    Kjør på medium hastighet i ca. \textbf{10 minutter}, og la deigen stå litt.
    
    \item 
    Tilsett så \textbf{2 ss} maltekstrakt og havregryn før du blander godt. 
    Nå kan du tilsette resten av vannet hvis ikke deigen er for bløt. 
    La deigen stå litt før du så tilsetter salt og blander i \textbf{10 minutter}.
    
    \item
    Ta deigen ut av bollen og bruk "strekk og brett"-metoden før du legger deigen tilbake i bollen. 
    Metoden går ut på å strekke et "hjørne" på deigen og brette det over deigen. Dette gjentas rundt hele deigen.
    
    La deigen modnes i bollen i \textbf{1 time} tildekket med plast.
    
    \item 
    Etter en time deler du deigen i tre, og lager rundvirk av brødemnene slik at de har en litt stram overflate. 
    La dem ligge en halvtime under plast. 
    Form brødemnene, og vet dem med vann før de dyppes i havregryn og legges i brødformene.
    
    \item
    Etter at brødene har stått under plast til de nesten har \textbf{doblet størrelsen} sin, skal de stekes. 
    Sett i ovnen på \textbf{240} grader og stek på nest \textbf{nederste} hylle. 
    Dette kommer an på ovnen – mange steker på nederste rille.
    
    Brødet skal stekes i \textbf{35-40 minutter}, men også dette kommer an på ovnen din. 
    
    \textit{\small Tips! 
    For å få en fin og tynn stekeskorpe på brødet kan man kaste inn \sfrac{1}{2} dl vann i bunnen av ovnen, deretter lukke døren raskt igjen. 
    Slå ned temperaturen til \textbf{220} grader, og ikke åpne ovnsdøren igjen før mot slutten av steketiden. 
    Dette er for å unngå å miste all dampen. 
    Maltekstraktet i oppskriften gjør at den tynne skorpen nærmest blir karamelisert.}
    
    \item
    Avkjøl brødene etter at du har tatt dem ut av ovnen. 
    Faktisk bør du ikke skjære i brødet før det er skikkelig avkjølt. 
    Hvis ikke siger mye av fuktigheten ut, og brødet blir fortere tørt.
\end{enumerate}