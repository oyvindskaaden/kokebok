\section*{Italiensk Pizzabunn (2 porsjoner)}

\subsection*{Dette trenger du}

\begin{table}[!htbp]
    %\centering
    \begin{tabular}{rl}
        \textbf{260 g}                  & pizzamel                              \\
        \textbf{4 g}                    & tørrgjær (\sfrac{1}{2} pose)          \\
        \textbf{\sfrac{1}{4} ts}        & sukker
    \end{tabular}
    \qquad
    \begin{tabular}{rl}
        \textbf{\sfrac{1}{2} ts}        & salt                                  \\
        \textbf{\sfrac{2}{3} ss}        & olivenolje                            \\
        \textbf{\SI{1.6}{\deci\liter}}  & vann (ca. \SI{20}{\celsius})
    \end{tabular}
\end{table}

\subsection*{Slik gjør du}

\begin{enumerate}
    \item 
    Ha mel, salt, sukker, gjær og vannet i bakebollen. Kjør blandingen i kjøkkenmaskin med eltekroker på lav til middels hastighet i ca. 10 minutter. Spe eventuelt med mer vann dersom deigen virker tørr, eller mer mel dersom den virker løs. 
    
    \item 
    Tilsett olivenolje og elt til du har en fin og smidig deig. 
    
    \item 
    Dekk bakebollen med lokk eller plast og la deigen heve i ca 40 minutter.
    
    \item 
    Del deigen opp i tre like store emner og bak de til boller. 
    La dette heve videre i ca 30 min eller mer. 
    
    \item 
    For å bake ut emnene er det lurt å ha tilstrekkelig med mel på benken. 
    Start med å flate den litt ut med hendene og begynn deretter fra midten og klem deigen sammen langs kantene.
    Resultatet er en relativt tynn bunn med en kant der luften fra hevingen fremdeles er bevart.

    \item
    Ha på det du selv ønsker av topping.
    Pizzaen stekes i øvre del av oven, så varmt det går. Helst på pizzasten eller pizzastål.
\end{enumerate}